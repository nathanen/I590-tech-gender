\documentclass[11pt]{article}
\usepackage{graphicx}
\usepackage[compact]{titlesec}

\usepackage{microtype}
\usepackage[text={5.5in,9.5in},centering]{geometry}

\usepackage{fontspec}
\defaultfontfeatures{Scale=MatchLowercase} 
\setmainfont[Mapping=tex-text]{Myriad Pro} 
\setmonofont[Mapping=tex-text]{Menlo}

% \usepackage[strict,backend=biber]{biblatex-chicago}
% \renewcommand{\cite}{\fullcite}

% \usepackage[citestyle=authortitle, natbib, backend=biber]{biblatex} 
% \renewcommand{\cite}{\autocite}
% \renewcommand\citep\autocite
\usepackage[authordate, backend=biber]{biblatex-chicago}
\renewcommand{\cite}{\textcite}


\bibliography{tech-gender}

\renewcommand \thesection{\Roman{section}}
\renewcommand \thesubsection{}
\renewcommand \thesubsubsection{}

\titleformat{\section}{\large \bf}{\thesection}{8pt}{}
\titleformat{\subsection}{\small \bf \it}{\thesubsection}{8pt}{}
\titleformat{\subsubsection}{\normalsize
  \bf}{\thesubsubsection}{8pt}{}

\titlespacing{\section}{0pt}{12pt}{2pt}
\titlespacing{\subsection}{0pt}{6pt}{2pt}
\titlespacing{\subsubsection}{-8pt}{*1}{8pt}

\newenvironment{reading} { \hangindent=1cm \begin{flushleft}
\begin{small} } { \end{small} \end{flushleft} }

\begin{document}
\thispagestyle{empty}
	
\begin{center}	
	
  \fontsize{28pt}{32pt} \selectfont {Technology \& Gender} \\
  \Large{INFO I-590 \textbullet \ \ Spring 2017}\\

\vspace{0.2in}

\setlength{\fboxsep}{0mm} %No padding in box
\setlength{\fboxrule}{2pt}

\end{center}
 
\vspace{0.2in}

In this seminar we will explore the literature on the history of gender and technology, with a particular focus on information technology.  From the ``computer girls'' of the early 20th century to the hyper-masculine culture of contemporary computing, ideas about gender have reflected and transformed our understanding of sexuality and gender.  Our goal in this seminar is to survey the best of the emerging literature on gender and computing, with an eye towards the practical application of gender theory into your future research projects.


\vspace{0.1in}

\begin{center} 
 Professor Nathan Ensmenger\\ \texttt{nensmeng@indiana.edu}\\
\vspace{0.1in}
Revision Date: \today
\end{center}

\vspace{0.3in}

\newpage

%\setlength{\parskip}{0.3cm}

\newpage
\large{\textbf{Course Schedule:}}\\


\noindent In addition to doing the required readings and preparing for discussions, you will be responsible for writing a short (1-2 pg) reading response paper each week.\\

\noindent The supplementary readings and extended bibliography are meant to make you aware of the larger literature, and to provide a guide for those of you who need further preparation for your qualifying exams or dissertation research.\\

\noindent A~note on books: all of the articles listed below will be made available electronically.  The books you are responsible for borrowing, purchasing, or otherwise acquiring.  I~did not order them via the bookstore, as in most cases you can find better bargains elsewhere.

\small
\let\realeverypar\everypar
\realeverypar{\the\myeverypar\the\everypar}% can add \youreverypar too!
\newtoks\everypar % LaTeX changes this to remove indentation etc.
\everypar{}
\newtoks\myeverypar \myeverypar{}

\myeverypar{\hangindent=1cm \small}

\section{January 11}

\fullcite{Turkle:1984hack}


\section{January 18}

\fullcite{Scott:1986ik}; \fullcite{Cowan:1976wda}; \fullcite{Oldenziel:1997to}

\subsection{Supplemental Readings}

For further development of her argument on the (lack of) industrialization in female domestic work, see \cite{Cowan:1983vm}.  \cite{Lerman:1997ui} provides an introduction to their influential special issue of \emph{Technology \& Culture} on women and technology in which the Oldenziel article above first appeared.


\section{January 25}

\fullcite{Bray:2007cb}; \fullcite{Kleif:2003wg}; \fullcite{Fischer:1988wl}

\subsection{Supplemental Readings}

\cite{Wajcman:2000vq}; \cite{Bray:1997wl}; \cite{Edwards:1990ua}; \cite{Pirsig:1974vs}; \cite{Ullman:1997vv}; \cite{Florman:1996um}


\section{February 1}

\fullcite{Oldenziel:1999vk}

\subsection{Supplemental Readings}
\cite{Maines:1999uw}; \cite{Rossiter:1982vn}, \cite{Tichi:1987wb}; \cite{Hacker:1989tm}

\section{February 8}

\fullcite{Strom:1992wx}

\subsection{Supplemental Readings}

\cite{Davies:1982vb} focuses on the typewriter as a gendered technology.  \cite{Milkman:1987tl} argues that women do not replace men during war; new and gendered positions are created for them. \cite{Benson:1987wl} describes women's work in retail. \cite{Tone:2001to} is a history of birth control, and includes a section on female entrepreneurs in this technology.

\section{February 15}

\fullcite{Hicks:2016uj} 

\subsection{Supplemental Readings}

Both \cite{Ensmenger:2010te} and \cite{Abbate:2012wq} cover the analogous story of the development of computer programming in the United States during this period.  \cite{Agar:2003wf}, while not explicitly about gender, provides the context for the turn towards ``machinic'' thinking in the British Civil Service.


\section{February 22}


\fullcite{Nakamura:2014gp}; \fullcite{Ensmenger:2010tc}; \fullcite{Haraway:1991uz}; \fullcite{Bernstein:1980wf}


\subsection{Supplemental Readings}

For other accounts of the roles that women played in the early computer industry, see \cite{Shetterly:2016vl}, \cite{Grier:2005tq}, and \cite{Gurer:1996it}.


\section{March 1}

\fullcite{Levy:1984ut}(selected excerpts); \fullcite{Eglash:2002wk}; \fullcite{Lagesen:2008vy}; \fullcite{Ensmenger:2015wx}

\subsection{Supplemental Readings}

\cite{Kidder:1981tj} was awarded the Pulitzer Prize for its gripping tale of computer engineers as Wild West heroes. \cite{Kocurek:2015cg} provides the larger context for understanding masculinity and video games.   \cite{Losse:2012um} updates these narratives for the Facebook era. If you have trouble understanding Haraway, read \cite{Gibson:1995un}, which covers some of the same territory in the form cyberpunk science fiction.  In fact, read the Gibson anyway.  It is beautiful, insightful, and powerfully influential in its own right.



\section{March 8}

\fullcite{Hayles:2008wq}

\subsection{Supplemental Readings}

For more on the relationship between embodiment and virtuality, see \cite{Stone:1996wp}.  In a foreshadowing of our section on queer computing, \cite{Wilson:2009wm} explores the ``confluence of sexual and intellectual matters'' that swirled around the tragic genius Walter Pitts, one of the key members of the early Cybernetics group. And in case you were not intrigued enough by last weeks's discussion to read Gibson's \emph{Neuromancer}, read it now.

\section{March 22}

\fullcite{Pascoe:2011cj}

\section{March 29}

\fullcite{Cohn:1993wq}; \fullcite{Nafus:2012gg}; \fullcite{Reagle:2012vr}


\section{April 5 }

\fullcite{Cassell:2000vw}

\subsection{Supplemental Readings}
\cite{Burrill2008}

\section{April 12}

\fullcite{Kafai:2008wl}

\subsection{Supplemental Readings}
\fullcite{Shaw:2015dr}


\section{April 19}

\fullcite{Gaboury:2015uv}; \fullcite{Nooney:2013vu}; \fullcite{Wu:2007gs}

\subsection{Supplemental Readings}
\cite{Ruberg:2017ww}

\section{April 26}

TBD.

\newpage
\section{Additional Resources}
\nocite{*} 
\printbibliography

\end{document}
