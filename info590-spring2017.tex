\documentclass[11pt]{article}
                  \usepackage{graphicx}
\usepackage[compact]{titlesec}

\usepackage{microtype}
\usepackage[text={5.5in,9.5in},centering]{geometry}

\usepackage{fontspec}
\defaultfontfeatures{Scale=MatchLowercase} 
\setmainfont[Mapping=tex-text]{Myriad Pro} 
\setmonofont[Mapping=tex-text]{Menlo}
% \usepackage[english]{babel}
% \usepackage[notes,short]{biblatex-chicago}
% \usepackage[style=apa, backend=biber]{biblatex} 

\usepackage[authordate, backend=biber]{biblatex-chicago}
\renewcommand{\cite}{\textcite}

\bibliography{tech-gender}

\renewcommand \thesection{\Roman{section}}
\renewcommand \thesubsection{}
\renewcommand \thesubsubsection{}

\titleformat{\section}{\large \bf}{\thesection}{8pt}{}
\titleformat{\subsection}{\small \bf \it}{\thesubsection}{8pt}{}
\titleformat{\subsubsection}{\normalsize
  \bf}{\thesubsubsection}{8pt}{}

\titlespacing{\section}{0pt}{12pt}{2pt}
\titlespacing{\subsection}{0pt}{6pt}{2pt}
\titlespacing{\subsubsection}{-8pt}{*1}{8pt}

\setlength{\parindent}{0in}

%  BEGIN: customize maketitle

\renewcommand{\maketitle}
{
\thispagestyle{empty}
	
\begin{center}	
	
  \fontsize{28pt}{32pt} \selectfont {Technology \& Gender} \\
  \Large{INFO I-590 \textbullet \ \ Spring 2017}\\

\vspace{0.2in}

\setlength{\fboxsep}{0mm} %No padding in box
\setlength{\fboxrule}{2pt}

\end{center}

}
%  END: customize maketitle

\author{Nathan Ensmenger}
\date{\today}
\title{Technology \& Gender}
\begin{document}

\maketitle
In this seminar we will explore the literature on the history of gender and technology, with a particular focus on information technology. From the "computer girls" of the early 20th century to the hyper-masculine culture of contemporary computing, developments in technology have reflected and transformed our understanding of sexuality and gender. Our goal in this seminar is to survey the best of the emerging literature on gender and computing, with an eye towards the practical application of gender theory into your future research projects.

\vspace{0.1in}

\begin{center} 
 Professor Nathan Ensmenger\\ \texttt{nensmeng@indiana.edu}\\
\vspace{0.1in}
Revision Date: \today
\end{center}

\newpage
\textbf{Course Schedule}\\

In addition to doing the required readings and preparing for discussions, you will be responsible for writing a short (1-2 pg)
reading response paper each week.\\

The supplementary readings and extended bibliography are meant to make you aware of the larger literature, and to provide a guide for those of you who need further preparation for your qualifying exams or
dissertation research.\\

A note on books: all of the articles listed below will be made available electronically. The books you are responsible for borrowing, purchasing, or otherwise acquiring. I did not order them via the bookstore, as in most cases you can find better bargains elsewhere.

\small
\let\realeverypar\everypar
\realeverypar{\the\myeverypar\the\everypar}
\newtoks\everypar
\everypar{}
\newtoks\myeverypar \myeverypar{}

% \myeverypar{\hangindent=1cm \small}

\section{January 11}

\fullcite{Turkle1984}


\subsection{Supplemental Readings}

\noindent The chapter on amateur radio operators in \cite{Douglas1987} provides a fascinating complement to the Turkle chapter on hackers that describes a similar phenomenon without reference to the uniquely immersive characteristics of electronic digital computers.  

\section{January 18}

\fullcite{Scott1986}; \fullcite{Cowan1976}; \fullcite{Oldenziel1997}

\subsection{Supplemental Readings}

For further development of her argument on the (lack of) industrialization in female domestic work, see \cite{Cowan1983}. \cite{Lerman1997} provides an introduction to their influential special issue of \emph{Technology \& Culture} on women and technology in which the Oldenziel article above first appeared.

\section{January 25}

\fullcite{Bray2007}; \fullcite{Kleif2003}; \fullcite{Fischer1988}

\subsection{Supplemental Readings}

\cite{Wajcman2000} provides a complementary overview of the historiography on technology and gender.  \cite{Bray1997} is Francesca Bray’s history of women and work in late Imperial China.   \cite{Florman1996} explores male pleasure in engineering, and \cite{Pirsig1974} is a literary/philosophical take on the same subject that has become something of a hacker bible.   \cite{Ullman1997} shows that taking pleasure in technology is not an exclusively male domain. 

\section{February 1}

\fullcite{Oldenziel1999}

\subsection{Supplemental Readings}

\cite{Rossiter1982} explores the history of women in science, with a particular emphasis on the relationship between professionalization and masculinization.   \cite{Tichi1987} provides a cultural and literary history of the engineer as a male hero. \cite{Hacker1989} explores the links between feminism, co-operatism and technology.  \cite{Maines1999} is a classic history of the vibrator.

\section{February 8}

\fullcite{Strom1992}

\subsection{Supplemental Readings}

\cite{Davies1982} focuses on the typewriter as a gendered technology.
\cite{Milkman1987} argues that women do not replace men during war;
new and gendered positions are created for them. \cite{Benson1987}
describes women's work in retail. \cite{Tone2001} is a history of
birth control, and includes a section on female entrepreneurs in this
technology.

\section{February 15}

\fullcite{Hicks2016}

\subsection{Supplemental Readings}

Both \cite{Ensmenger2010} and \cite{Abbate2012} cover the
analogous story of the development of computer programming in the United
States during this period. \cite{Agar2003}, while not explicitly
about gender, provides the context for the turn towards "machinic"
thinking in the British Civil Service. \cite{Edwards1990} is a pioneering work on the gendering of computer programming, and \cite{Light1999} traces the erasure of the ENIAC women from the history of computing.


\section{February 22}

\fullcite{Nakamura2014}; \fullcite{Ensmenger2010}; \fullcite{Haraway1991};
\fullcite{Bernstein1980}

\subsection{Supplemental Readings}

For other accounts of the roles that women played in the early computer industry, see \cite{Shetterly2016}, \cite{Grier2005}, and \cite{Gurer2002}.

\section{March 1}

\fullcite{Levy1984} (selected excerpts); \fullcite{Eglash2002}
\fullcite{Lagesen2008}; \fullcite{Ensmenger2015}

\subsection{Supplemental Readings}

\cite{Kidder1981} was awarded the Pulitzer Prize for its gripping
tale of computer engineers as Wild West heroes. \cite{Kocurek2015}
provides the larger context for understanding masculinity and video
games. \cite{Losse2012} updates these narratives for the Facebook
era. If you have trouble understanding Haraway, read
\cite{Gibson1995}, which covers some of the same territory in the
form cyberpunk science fiction. In fact, read the Gibson anyway. It is
beautiful, insightful, and powerfully influential in its own right.

\section{March 8}

\fullcite{Hayles2008}

\subsection{Supplemental Readings}

For more on the relationship between embodiment and virtuality, see \cite{Stone1996} and \cite{Balsamo1996}.  For an overview of the history of cybernetics and its relationship to contemporary information technology, see \cite{Kline2015}.  For an interesting foreshadowing of our section on queer computing, \cite{Wilson2009} explores the ``confluence of sexual and intellectual matters'' that swirled around the tragic genius Walter Pitts, one of the key members of the early Cybernetics group. And in case you were not intrigued enough by last weeks's discussion to read Gibson's \emph{Neuromancer}, read it now.

\section{March 22}

\fullcite{Pascoe2011}

\subsection{Supplemental Readings}

For a more general history of American masculinity, see \cite{Rotundo1994}. \cite{Mellstrom2004} focuses more specifically on the role of technology in shaping masculine norms, and \cite{Burrill2008} even more specifically on the performance of masculinity in video game culture.

\section{March 29}

\fullcite{Cohn1993}; \fullcite{Nafus2012}; \fullcite{Adam2003}; \fullcite{Beran2017}


\subsection{Supplemental Readings}
For a brief history of games in the contest of military defense intellectuals, see \cite{Ghamari-Tabrizi2000} and \cite{JenniferLight2008}. \cite{Mead2013} explores the use of video games as a recruitment, training, and therapeutic tool within the United States military.

\section{April 5}

\fullcite{Cassell2000}

\subsection{Supplemental Readings}

\cite{Salter2012} discusses the "hypermasculine" practices of video game culture.  \cite{Bardzell2010} outlines a feminist approach to computer interface design. \cite{Varney2002} explores the history of masculine toys for boys.  

\section{April 12}

\fullcite{Kafai2008}

\subsection{Supplemental Readings}

\cite{Shaw2015} argues that gamers necessarily experience the intersection of race, gender, and sexuality. \cite{Kocurek2015} situates video game culuter in the larger history of video game arcades.

\section{April 19}

\fullcite{Gaboury2015}; \fullcite{Nooney2013}; \fullcite{Wu2007}; \cite{Gray2014}

\subsection{Supplemental Readings}

\cite{Ruberg2017} is an anthology of essays on queer game studies.

\section{April 26}

TBD.

\newpage

What follows is a list of relevant resources related to gender and computing. It includes the full citation information for all of the readings listed in the syllabus above, but also many supplemental materials.\\

In addition, you might find useful the list of women and gender non-conforming people writing about technology found at \url{https://goo.gl/m6J2dm}.\\


\nocite{*} 
\printbibliography
\end{document}
